% Options for packages loaded elsewhere
\PassOptionsToPackage{unicode}{hyperref}
\PassOptionsToPackage{hyphens}{url}
%
\documentclass[
]{article}
\usepackage{amsmath,amssymb}
\usepackage{iftex}
\ifPDFTeX
  \usepackage[T1]{fontenc}
  \usepackage[utf8]{inputenc}
  \usepackage{textcomp} % provide euro and other symbols
\else % if luatex or xetex
  \usepackage{unicode-math} % this also loads fontspec
  \defaultfontfeatures{Scale=MatchLowercase}
  \defaultfontfeatures[\rmfamily]{Ligatures=TeX,Scale=1}
\fi
\usepackage{lmodern}
\ifPDFTeX\else
  % xetex/luatex font selection
\fi
% Use upquote if available, for straight quotes in verbatim environments
\IfFileExists{upquote.sty}{\usepackage{upquote}}{}
\IfFileExists{microtype.sty}{% use microtype if available
  \usepackage[]{microtype}
  \UseMicrotypeSet[protrusion]{basicmath} % disable protrusion for tt fonts
}{}
\makeatletter
\@ifundefined{KOMAClassName}{% if non-KOMA class
  \IfFileExists{parskip.sty}{%
    \usepackage{parskip}
  }{% else
    \setlength{\parindent}{0pt}
    \setlength{\parskip}{6pt plus 2pt minus 1pt}}
}{% if KOMA class
  \KOMAoptions{parskip=half}}
\makeatother
\usepackage{xcolor}
\usepackage[margin=1in]{geometry}
\usepackage{color}
\usepackage{fancyvrb}
\newcommand{\VerbBar}{|}
\newcommand{\VERB}{\Verb[commandchars=\\\{\}]}
\DefineVerbatimEnvironment{Highlighting}{Verbatim}{commandchars=\\\{\}}
% Add ',fontsize=\small' for more characters per line
\usepackage{framed}
\definecolor{shadecolor}{RGB}{248,248,248}
\newenvironment{Shaded}{\begin{snugshade}}{\end{snugshade}}
\newcommand{\AlertTok}[1]{\textcolor[rgb]{0.94,0.16,0.16}{#1}}
\newcommand{\AnnotationTok}[1]{\textcolor[rgb]{0.56,0.35,0.01}{\textbf{\textit{#1}}}}
\newcommand{\AttributeTok}[1]{\textcolor[rgb]{0.13,0.29,0.53}{#1}}
\newcommand{\BaseNTok}[1]{\textcolor[rgb]{0.00,0.00,0.81}{#1}}
\newcommand{\BuiltInTok}[1]{#1}
\newcommand{\CharTok}[1]{\textcolor[rgb]{0.31,0.60,0.02}{#1}}
\newcommand{\CommentTok}[1]{\textcolor[rgb]{0.56,0.35,0.01}{\textit{#1}}}
\newcommand{\CommentVarTok}[1]{\textcolor[rgb]{0.56,0.35,0.01}{\textbf{\textit{#1}}}}
\newcommand{\ConstantTok}[1]{\textcolor[rgb]{0.56,0.35,0.01}{#1}}
\newcommand{\ControlFlowTok}[1]{\textcolor[rgb]{0.13,0.29,0.53}{\textbf{#1}}}
\newcommand{\DataTypeTok}[1]{\textcolor[rgb]{0.13,0.29,0.53}{#1}}
\newcommand{\DecValTok}[1]{\textcolor[rgb]{0.00,0.00,0.81}{#1}}
\newcommand{\DocumentationTok}[1]{\textcolor[rgb]{0.56,0.35,0.01}{\textbf{\textit{#1}}}}
\newcommand{\ErrorTok}[1]{\textcolor[rgb]{0.64,0.00,0.00}{\textbf{#1}}}
\newcommand{\ExtensionTok}[1]{#1}
\newcommand{\FloatTok}[1]{\textcolor[rgb]{0.00,0.00,0.81}{#1}}
\newcommand{\FunctionTok}[1]{\textcolor[rgb]{0.13,0.29,0.53}{\textbf{#1}}}
\newcommand{\ImportTok}[1]{#1}
\newcommand{\InformationTok}[1]{\textcolor[rgb]{0.56,0.35,0.01}{\textbf{\textit{#1}}}}
\newcommand{\KeywordTok}[1]{\textcolor[rgb]{0.13,0.29,0.53}{\textbf{#1}}}
\newcommand{\NormalTok}[1]{#1}
\newcommand{\OperatorTok}[1]{\textcolor[rgb]{0.81,0.36,0.00}{\textbf{#1}}}
\newcommand{\OtherTok}[1]{\textcolor[rgb]{0.56,0.35,0.01}{#1}}
\newcommand{\PreprocessorTok}[1]{\textcolor[rgb]{0.56,0.35,0.01}{\textit{#1}}}
\newcommand{\RegionMarkerTok}[1]{#1}
\newcommand{\SpecialCharTok}[1]{\textcolor[rgb]{0.81,0.36,0.00}{\textbf{#1}}}
\newcommand{\SpecialStringTok}[1]{\textcolor[rgb]{0.31,0.60,0.02}{#1}}
\newcommand{\StringTok}[1]{\textcolor[rgb]{0.31,0.60,0.02}{#1}}
\newcommand{\VariableTok}[1]{\textcolor[rgb]{0.00,0.00,0.00}{#1}}
\newcommand{\VerbatimStringTok}[1]{\textcolor[rgb]{0.31,0.60,0.02}{#1}}
\newcommand{\WarningTok}[1]{\textcolor[rgb]{0.56,0.35,0.01}{\textbf{\textit{#1}}}}
\usepackage{graphicx}
\makeatletter
\def\maxwidth{\ifdim\Gin@nat@width>\linewidth\linewidth\else\Gin@nat@width\fi}
\def\maxheight{\ifdim\Gin@nat@height>\textheight\textheight\else\Gin@nat@height\fi}
\makeatother
% Scale images if necessary, so that they will not overflow the page
% margins by default, and it is still possible to overwrite the defaults
% using explicit options in \includegraphics[width, height, ...]{}
\setkeys{Gin}{width=\maxwidth,height=\maxheight,keepaspectratio}
% Set default figure placement to htbp
\makeatletter
\def\fps@figure{htbp}
\makeatother
\setlength{\emergencystretch}{3em} % prevent overfull lines
\providecommand{\tightlist}{%
  \setlength{\itemsep}{0pt}\setlength{\parskip}{0pt}}
\setcounter{secnumdepth}{-\maxdimen} % remove section numbering
\ifLuaTeX
  \usepackage{selnolig}  % disable illegal ligatures
\fi
\IfFileExists{bookmark.sty}{\usepackage{bookmark}}{\usepackage{hyperref}}
\IfFileExists{xurl.sty}{\usepackage{xurl}}{} % add URL line breaks if available
\urlstyle{same}
\hypersetup{
  pdftitle={Data Report},
  pdfauthor={Hannah Schweren},
  hidelinks,
  pdfcreator={LaTeX via pandoc}}

\title{Data Report}
\author{Hannah Schweren}
\date{2024-01-24}

\begin{document}
\maketitle

\begin{Shaded}
\begin{Highlighting}[]
\CommentTok{\#load necessary packages}

\FunctionTok{library}\NormalTok{(rvest)}
\FunctionTok{library}\NormalTok{(dplyr)}
\end{Highlighting}
\end{Shaded}

\begin{verbatim}
## 
## Attaching package: 'dplyr'
\end{verbatim}

\begin{verbatim}
## The following objects are masked from 'package:stats':
## 
##     filter, lag
\end{verbatim}

\begin{verbatim}
## The following objects are masked from 'package:base':
## 
##     intersect, setdiff, setequal, union
\end{verbatim}

\begin{Shaded}
\begin{Highlighting}[]
\FunctionTok{library}\NormalTok{(stringr) }\CommentTok{\#für word count}
\FunctionTok{library}\NormalTok{(ggplot2)}
\FunctionTok{library}\NormalTok{(legislatoR)}
\end{Highlighting}
\end{Shaded}

\hypertarget{data-import}{%
\section{Data import}\label{data-import}}

After aquiring the data from the CLD containing the ``wikititle'', I can
use this to scrape the Wikipedia articles. I subsetted the data, only
focussing on politicians that are still alive (death = NA) as I consider
this more relevant for my analysis.

\begin{Shaded}
\begin{Highlighting}[]
\CommentTok{\#import raw data }

\CommentTok{\#raw data was aquired using the following code (using France as an example)}
\CommentTok{\# fr\_core \textless{}{-}  get\_core((legislature = "fra"))}
\CommentTok{\# fr\_core\_alive \textless{}{-} fr\_core \%\textgreater{}\% filter(is.na(death))}

\DocumentationTok{\#\#\#text aquisition german}

\CommentTok{\# de\_text\_pipeline \textless{}{-} function(page\_name) \{}
\CommentTok{\#   Sys.sleep(runif(1, 1, 2))}
\CommentTok{\#   }
\CommentTok{\#   \# Check if page\_name is missing}
\CommentTok{\#   if (is.na(page\_name) || page\_name == "") \{}
\CommentTok{\#     return("No Wikipedia page name provided or missing.")}
\CommentTok{\#   \}}
\CommentTok{\#   }
\CommentTok{\#   \# Try fetching Wikipedia content}
\CommentTok{\#   tryCatch(\{}
\CommentTok{\#     wp\_content \textless{}{-} WikipediR::page\_content("de", "wikipedia", page\_name = page\_name)}
\CommentTok{\#     plain\_text \textless{}{-} html\_text(read\_html(wp\_content$parse$text$\textasciigrave{}*\textasciigrave{}))}
\CommentTok{\#     return(plain\_text)}
\CommentTok{\#   \}, error = function(e) \{}
\CommentTok{\#     return(paste("Error fetching content for page:", page\_name))}
\CommentTok{\#   \})}
\CommentTok{\# \}}


\NormalTok{cze\_alive\_text }\OtherTok{\textless{}{-}} \FunctionTok{read.csv}\NormalTok{(}\StringTok{"raw\_data/cze\_alive\_text.csv"}\NormalTok{)}

\NormalTok{deu\_alive\_text }\OtherTok{\textless{}{-}} \FunctionTok{read.csv}\NormalTok{(}\StringTok{"raw\_data/deu\_alive\_text.csv"}\NormalTok{)}

\NormalTok{fr\_alive\_text }\OtherTok{\textless{}{-}} \FunctionTok{read.csv}\NormalTok{(}\StringTok{"raw\_data/fr\_alive\_text.csv"}\NormalTok{)}

\NormalTok{usa\_alive\_text }\OtherTok{\textless{}{-}} \FunctionTok{read.csv}\NormalTok{(}\StringTok{"raw\_data/usa\_alive\_text.csv"}\NormalTok{)}

\NormalTok{gbr\_alive\_text }\OtherTok{\textless{}{-}} \FunctionTok{read.csv}\NormalTok{(}\StringTok{"raw\_data/gbr\_alive\_text.csv"}\NormalTok{)}

\NormalTok{sco\_alive\_text }\OtherTok{\textless{}{-}} \FunctionTok{read.csv}\NormalTok{(}\StringTok{"raw\_data/sco\_alive\_text.csv"}\NormalTok{)}

\NormalTok{irl\_alive\_text }\OtherTok{\textless{}{-}} \FunctionTok{read.csv}\NormalTok{(}\StringTok{"raw\_data/irl\_alive\_text.csv"}\NormalTok{)}

\NormalTok{esp\_alive\_text }\OtherTok{\textless{}{-}} \FunctionTok{read.csv}\NormalTok{(}\StringTok{"raw\_data/esp\_alive\_text.csv"}\NormalTok{)}

\NormalTok{can\_alive\_text }\OtherTok{\textless{}{-}} \FunctionTok{read.csv}\NormalTok{(}\StringTok{"raw\_data/can\_alive\_text.csv"}\NormalTok{)}

\NormalTok{aut\_alive\_text }\OtherTok{\textless{}{-}} \FunctionTok{read.csv}\NormalTok{(}\StringTok{"raw\_data/aut\_alive\_text.csv"}\NormalTok{) }
\end{Highlighting}
\end{Shaded}

\hypertarget{functions}{%
\section{functions}\label{functions}}

\begin{Shaded}
\begin{Highlighting}[]
\CommentTok{\# Necessary functions for the following data preprocessing}

\NormalTok{clean\_data }\OtherTok{\textless{}{-}} \ControlFlowTok{function}\NormalTok{(df) \{}
\NormalTok{  initial\_rows }\OtherTok{\textless{}{-}} \FunctionTok{nrow}\NormalTok{(df)}
  
  \CommentTok{\# Remove CSS{-}like structures}
\NormalTok{  df}\SpecialCharTok{$}\NormalTok{plain\_text }\OtherTok{\textless{}{-}} \FunctionTok{str\_remove\_all}\NormalTok{(df}\SpecialCharTok{$}\NormalTok{plain\_text, }\StringTok{"}\SpecialCharTok{\textbackslash{}\textbackslash{}}\StringTok{..*?}\SpecialCharTok{\textbackslash{}\textbackslash{}}\StringTok{\{.*?}\SpecialCharTok{\textbackslash{}\textbackslash{}}\StringTok{\}"}\NormalTok{)}
  
  \CommentTok{\# Initialize counters for removal reasons}
\NormalTok{  removal\_reason\_redirect }\OtherTok{\textless{}{-}} \FunctionTok{sum}\NormalTok{(}\FunctionTok{grepl}\NormalTok{(}\StringTok{"\^{}(Redirect to:|Weiterleitung nach:|Rediriger vers:|Redirige a:|Přesměrování na:)"}\NormalTok{, df}\SpecialCharTok{$}\NormalTok{plain\_text, }\AttributeTok{ignore.case =} \ConstantTok{TRUE}\NormalTok{))}
\NormalTok{  removal\_reason\_refering\_page }\OtherTok{\textless{}{-}} \FunctionTok{sum}\NormalTok{(}\FunctionTok{grepl}\NormalTok{(}\StringTok{"may refer to:|ist der Name folgender Personen:|Cette page d\textquotesingle{}homonymie répertorie différentes personnes|může být:"}\NormalTok{, df}\SpecialCharTok{$}\NormalTok{plain\_text, }\AttributeTok{ignore.case =} \ConstantTok{TRUE}\NormalTok{))}
\NormalTok{  removal\_reason\_not\_found }\OtherTok{\textless{}{-}} \FunctionTok{sum}\NormalTok{(}\FunctionTok{grepl}\NormalTok{(}\StringTok{"\^{}(Error fetching content for page:|No Wikipedia page name provided or missing|Es wurde kein Wikipedia{-}Seitenname angegeben)"}\NormalTok{, df}\SpecialCharTok{$}\NormalTok{plain\_text, }\AttributeTok{ignore.case =} \ConstantTok{TRUE}\NormalTok{))}
  
  
  \CommentTok{\# Filter rows based on specific conditions}
\NormalTok{  df }\OtherTok{\textless{}{-}}\NormalTok{ df }\SpecialCharTok{\%\textgreater{}\%}
    \FunctionTok{filter}\NormalTok{(}\SpecialCharTok{!}\FunctionTok{grepl}\NormalTok{(}\StringTok{"\^{}(Redirect to:|Weiterleitung nach:|Rediriger vers:|Redirige a:|Přesměrování na:)"}\NormalTok{, plain\_text, }\AttributeTok{ignore.case =} \ConstantTok{TRUE}\NormalTok{) }\SpecialCharTok{\&}
             \SpecialCharTok{!}\FunctionTok{grepl}\NormalTok{(}\StringTok{"may refer to:|ist der Name folgender Personen:|Cette page d\textquotesingle{}homonymie répertorie différentes personnes|může být:"}\NormalTok{, plain\_text, }\AttributeTok{ignore.case =} \ConstantTok{TRUE}\NormalTok{) }\SpecialCharTok{\&}
             \SpecialCharTok{!}\FunctionTok{grepl}\NormalTok{(}\StringTok{"Error fetching content for page:|No Wikipedia page name provided or missing|Es wurde kein Wikipedia{-}Seitenname angegeben"}\NormalTok{, plain\_text, }\AttributeTok{ignore.case =} \ConstantTok{TRUE}\NormalTok{))}
  
  \CommentTok{\# Calculate the number of rows removed}
\NormalTok{  rows\_removed }\OtherTok{\textless{}{-}}\NormalTok{ initial\_rows }\SpecialCharTok{{-}} \FunctionTok{nrow}\NormalTok{(df)}
  
  \CommentTok{\# Print statistics about the removal reasons}
  \FunctionTok{cat}\NormalTok{(}\StringTok{"Removal reasons:}\SpecialCharTok{\textbackslash{}n}\StringTok{"}\NormalTok{)}
  \FunctionTok{cat}\NormalTok{(}\StringTok{"  {-} Redirect:"}\NormalTok{, removal\_reason\_redirect, }\StringTok{"}\SpecialCharTok{\textbackslash{}n}\StringTok{"}\NormalTok{)}
  \FunctionTok{cat}\NormalTok{(}\StringTok{"  {-} Reference Page:"}\NormalTok{, removal\_reason\_refering\_page, }\StringTok{"}\SpecialCharTok{\textbackslash{}n}\StringTok{"}\NormalTok{)}
  \FunctionTok{cat}\NormalTok{(}\StringTok{"  {-} Not Found/no name\_provided:"}\NormalTok{, removal\_reason\_not\_found, }\StringTok{"}\SpecialCharTok{\textbackslash{}n}\StringTok{"}\NormalTok{)}
  
  
  \CommentTok{\# Create a message about the cleaning process}
  \FunctionTok{cat}\NormalTok{(}\StringTok{"Cleaned data: Removed"}\NormalTok{, rows\_removed, }\StringTok{"rows.}\SpecialCharTok{\textbackslash{}n}\StringTok{"}\NormalTok{)}
  
  \CommentTok{\# Return the cleaned data frame}
  \FunctionTok{return}\NormalTok{(df)}
\NormalTok{\}}


\NormalTok{traffic\_metrics }\OtherTok{\textless{}{-}} \ControlFlowTok{function}\NormalTok{(traffic\_data) \{}
  \CommentTok{\# Format the date}
\NormalTok{  traffic\_data}\SpecialCharTok{$}\NormalTok{date }\OtherTok{\textless{}{-}} \FunctionTok{format}\NormalTok{(traffic\_data}\SpecialCharTok{$}\NormalTok{date, }\StringTok{"\%Y{-}\%m"}\NormalTok{)}
  
  \CommentTok{\# Total per politician}
\NormalTok{  total\_traffic\_per\_politician }\OtherTok{\textless{}{-}}\NormalTok{ traffic\_data }\SpecialCharTok{\%\textgreater{}\%}
    \FunctionTok{group\_by}\NormalTok{(pageid) }\SpecialCharTok{\%\textgreater{}\%}
    \FunctionTok{summarise}\NormalTok{(}\AttributeTok{total\_traffic =} \FunctionTok{sum}\NormalTok{(traffic))}
  
  \CommentTok{\# Average per month per politician}
\NormalTok{  average\_traffic\_per\_politician }\OtherTok{\textless{}{-}}\NormalTok{ total\_traffic\_per\_politician }\SpecialCharTok{\%\textgreater{}\%}
    \FunctionTok{mutate}\NormalTok{(}\AttributeTok{average\_traffic =}\NormalTok{ total\_traffic }\SpecialCharTok{/} \FunctionTok{n\_distinct}\NormalTok{(traffic\_data}\SpecialCharTok{$}\NormalTok{date))}
  
  \CommentTok{\# Convert pageid to numeric}
\NormalTok{  average\_traffic\_per\_politician}\SpecialCharTok{$}\NormalTok{pageid }\OtherTok{\textless{}{-}} \FunctionTok{as.numeric}\NormalTok{(average\_traffic\_per\_politician}\SpecialCharTok{$}\NormalTok{pageid)}
  
  \CommentTok{\# Return the result}
  \FunctionTok{return}\NormalTok{(average\_traffic\_per\_politician)}
\NormalTok{\}}


\NormalTok{count\_words }\OtherTok{\textless{}{-}} \ControlFlowTok{function}\NormalTok{(text) \{}
\NormalTok{  words }\OtherTok{\textless{}{-}} \FunctionTok{str\_extract\_all}\NormalTok{(text, }\StringTok{"}\SpecialCharTok{\textbackslash{}\textbackslash{}}\StringTok{b}\SpecialCharTok{\textbackslash{}\textbackslash{}}\StringTok{w+}\SpecialCharTok{\textbackslash{}\textbackslash{}}\StringTok{b"}\NormalTok{)[[}\DecValTok{1}\NormalTok{]]}
  \FunctionTok{return}\NormalTok{(}\FunctionTok{length}\NormalTok{(words))}
\NormalTok{\}}
\end{Highlighting}
\end{Shaded}

the function to clean the data removes unreadable parts of the html
format and leaves us with human readable text of the politician's
Wikipedia article. Further, it removes datapoints that weren't
succesfully retreiving an artcile for reasons of redirects (name
changes), missing Wikipedia pages (``wiki\_title'') or references pages
(``may refer to\ldots{}'').

\hypertarget{data-cleaning}{%
\section{data cleaning}\label{data-cleaning}}

\begin{Shaded}
\begin{Highlighting}[]
\NormalTok{cze }\OtherTok{\textless{}{-}} \FunctionTok{clean\_data}\NormalTok{(cze\_alive\_text)}
\end{Highlighting}
\end{Shaded}

\begin{verbatim}
## Removal reasons:
##   - Redirect: 9 
##   - Reference Page: 3 
##   - Not Found/no name_provided: 1 
## Cleaned data: Removed 13 rows.
\end{verbatim}

\begin{Shaded}
\begin{Highlighting}[]
\NormalTok{fra }\OtherTok{\textless{}{-}} \FunctionTok{clean\_data}\NormalTok{(fr\_alive\_text)}
\end{Highlighting}
\end{Shaded}

\begin{verbatim}
## Removal reasons:
##   - Redirect: 0 
##   - Reference Page: 6 
##   - Not Found/no name_provided: 1 
## Cleaned data: Removed 7 rows.
\end{verbatim}

\begin{Shaded}
\begin{Highlighting}[]
\NormalTok{deu }\OtherTok{\textless{}{-}} \FunctionTok{clean\_data}\NormalTok{(deu\_alive\_text)}
\end{Highlighting}
\end{Shaded}

\begin{verbatim}
## Removal reasons:
##   - Redirect: 11 
##   - Reference Page: 27 
##   - Not Found/no name_provided: 4 
## Cleaned data: Removed 42 rows.
\end{verbatim}

\begin{Shaded}
\begin{Highlighting}[]
\NormalTok{usa }\OtherTok{\textless{}{-}} \FunctionTok{clean\_data}\NormalTok{(usa\_alive\_text)}
\end{Highlighting}
\end{Shaded}

\begin{verbatim}
## Removal reasons:
##   - Redirect: 103 
##   - Reference Page: 14 
##   - Not Found/no name_provided: 0 
## Cleaned data: Removed 117 rows.
\end{verbatim}

\begin{Shaded}
\begin{Highlighting}[]
\NormalTok{gbr }\OtherTok{\textless{}{-}} \FunctionTok{clean\_data}\NormalTok{(gbr\_alive\_text)}
\end{Highlighting}
\end{Shaded}

\begin{verbatim}
## Removal reasons:
##   - Redirect: 55 
##   - Reference Page: 19 
##   - Not Found/no name_provided: 1598 
## Cleaned data: Removed 1672 rows.
\end{verbatim}

\begin{Shaded}
\begin{Highlighting}[]
\NormalTok{irl }\OtherTok{\textless{}{-}} \FunctionTok{clean\_data}\NormalTok{(irl\_alive\_text)}
\end{Highlighting}
\end{Shaded}

\begin{verbatim}
## Removal reasons:
##   - Redirect: 19 
##   - Reference Page: 19 
##   - Not Found/no name_provided: 0 
## Cleaned data: Removed 38 rows.
\end{verbatim}

\begin{Shaded}
\begin{Highlighting}[]
\NormalTok{sco }\OtherTok{\textless{}{-}} \FunctionTok{clean\_data}\NormalTok{(sco\_alive\_text)}
\end{Highlighting}
\end{Shaded}

\begin{verbatim}
## Removal reasons:
##   - Redirect: 3 
##   - Reference Page: 1 
##   - Not Found/no name_provided: 0 
## Cleaned data: Removed 4 rows.
\end{verbatim}

\begin{Shaded}
\begin{Highlighting}[]
\NormalTok{esp }\OtherTok{\textless{}{-}} \FunctionTok{clean\_data}\NormalTok{(esp\_alive\_text)}
\end{Highlighting}
\end{Shaded}

\begin{verbatim}
## Removal reasons:
##   - Redirect: 28 
##   - Reference Page: 0 
##   - Not Found/no name_provided: 1057 
## Cleaned data: Removed 1085 rows.
\end{verbatim}

\begin{Shaded}
\begin{Highlighting}[]
\NormalTok{aut }\OtherTok{\textless{}{-}} \FunctionTok{clean\_data}\NormalTok{(aut\_alive\_text)}
\end{Highlighting}
\end{Shaded}

\begin{verbatim}
## Removal reasons:
##   - Redirect: 7 
##   - Reference Page: 14 
##   - Not Found/no name_provided: 5 
## Cleaned data: Removed 26 rows.
\end{verbatim}

\begin{Shaded}
\begin{Highlighting}[]
\NormalTok{can }\OtherTok{\textless{}{-}} \FunctionTok{clean\_data}\NormalTok{(can\_alive\_text)}
\end{Highlighting}
\end{Shaded}

\begin{verbatim}
## Removal reasons:
##   - Redirect: 25 
##   - Reference Page: 13 
##   - Not Found/no name_provided: 0 
## Cleaned data: Removed 38 rows.
\end{verbatim}

I am currently using the ``wikititle'' as variable to use the API, which
leads to some problems with redirects, when politicans changed their
name after the creation of the CLD. Unfortunately, scraping via the
``pageid'' did not work. The number of redirects seems still acceptable
for me. A bigger problem is raised in the british and spanish data,
where a lot of missing data occurs because of the page not being found.
This is due to missing ``wikititle'' and ``pageid'', which means that
these politicians don't have a unique Wikipedia page or Wikidata ID. To
be discussed if this poses a problem for the further analysis of these
countries' politicians.

\hypertarget{data-exploration}{%
\section{data exploration}\label{data-exploration}}

\begin{Shaded}
\begin{Highlighting}[]
\CommentTok{\#combine all data in one df}

\NormalTok{all\_countries }\OtherTok{\textless{}{-}} \FunctionTok{rbind}\NormalTok{(deu, cze, fra, usa, sco, irl, can, aut, esp, gbr)}
\NormalTok{all\_countries }\OtherTok{\textless{}{-}}\NormalTok{ all\_countries}\SpecialCharTok{\%\textgreater{}\%}
  \FunctionTok{filter}\NormalTok{(}\SpecialCharTok{!}\FunctionTok{is.na}\NormalTok{(sex))}
\end{Highlighting}
\end{Shaded}

First, let's have a look at the dataset, containing all countries and
see how it is setup.

\begin{Shaded}
\begin{Highlighting}[]
\FunctionTok{head}\NormalTok{(all\_countries)}
\end{Highlighting}
\end{Shaded}

\begin{verbatim}
##   country   pageid wikidataid         wikititle                 name    sex
## 1     DEU   174000    Q340387    Achim_Großmann       Achim Großmann   male
## 2     DEU  9980355  Q39678866     Achim_Kessler        Achim Kessler   male
## 3     DEU  5166669    Q340448        Achim_Post           Achim Post   male
## 4     DEU   261258    Q354647 Adelheid_Tröscher Adelheid D. Tröscher female
## 5     DEU 11924573  Q96096131    Adis_Ahmetovic       Adis Ahmetovic   male
## 6     DEU  3314903    Q363517    Adolf_Ostertag       Adolf Ostertag   male
##   ethnicity               religion      birth death        birthplace
## 1      <NA>            catholicism 1947-04-17    NA  50.77621,6.08379
## 2     white                   <NA> 1964-08-02    NA  48.12472,8.33083
## 3     white protestantism lutheran 1959-05-02    NA  52.41667,8.61667
## 4      <NA>                   <NA> 1939-02-16    NA 52.51667,13.38333
## 5      <NA>                   <NA> 1993-07-27    NA  52.37444,9.73861
## 6      <NA>                   <NA> 1939-07-22    NA   49.3269,11.0631
##   deathplace
## 1       <NA>
## 2       <NA>
## 3       <NA>
## 4       <NA>
## 5       <NA>
## 6       <NA>
##                                                                                                                                                                                                                                                                                                                                                                                                                                                                                                                                                                                                                                                                                                                                                                                                                                                                                                                                                                                                                                                                                                                                                                                                                                                                                                                                                                                                                                                                                                                                                                                                                                                                                                                                                                                                                                                                                                                                                                                                                                                                                                                                                                                                                                                                                                                                                                                                                                                                                                                                                                                                                                                                                                                                                                                                                                                                                                                                                                                                                                                                                                                                                                                                                                                                                                                                                                                                                                                                                                                                                                                                                                                                                                                                                                                                                                                                                                                                                                                                                                                                                                                                                                                                                                                                                                                                                                                                                                                                                                                                                                                                                                                                                                                                                                                                                                                                                                                                                                                                                                                                                                                                                                                                                                                                                                                                                                                                                                                                                                                                                                                                                                                                                                                                                                                                                                                                                                                                                                                                                                                                                                                                                                                                                                                                                                                                                                                                                                                                                                                    plain_text
## 1                                                                                                                                                                                                                                                                                                                                                                                                                                                                                                                                                                                                                                                                                                                                                                                                                                                                                                                                                                                                                                                                                                                                                                                                                                                                                                                                                                                                                                                                                                                                                                                                                                                                                                                                                                                                                                                                                                                                                                                                                                                                                                                                                                                                                                                          Achim Großmann (* 17. April 1947 in Aachen; † 14. April 2023 in Würselen[1]) war ein deutscher Politiker (SPD). Er war von 1998 bis 2009 Parlamentarischer Staatssekretär beim Bundesminister für Verkehr, Bau- und Wohnungswesen bzw. ab 2005 Bundesminister für Verkehr, Bau und Stadtentwicklung.\n\nInhaltsverzeichnis\n1 Leben und Beruf\n2 Partei\n3 Abgeordneter\n4 Öffentliche Ämter\n5 Kabinette\n6 Veröffentlichungen\n7 Auszeichnungen\n8 Weblinks\n9 Einzelnachweise\n\n\nLeben und Beruf[Bearbeiten | Quelltext bearbeiten]\nNach dem Abitur 1966 am Kaiser-Karls-Gymnasium in Aachen absolvierte Großmann ein Studium der Psychologie an der TH Aachen, welches er 1972 als Diplom-Psychologe beendete. Er war dann bis 1986 als Erziehungsberater an der Beratungsstelle für Eltern, Kinder und Jugendliche in Alsdorf tätig, seit 1979 als deren Leiter. Parallel dazu hat er einige Semester als Dozent für Verwaltungspsychologie an der Fachhochschule für öffentliche Verwaltung in Aachen gearbeitet.\nAchim Großmann war geschieden und hatte zwei Kinder.\n\nPartei[Bearbeiten | Quelltext bearbeiten]\nSeit 1971 war er Mitglied der SPD. Von 1982 bis 1996 war er Vorsitzender des SPD-Unterbezirks Kreis Aachen und gehörte von 1983 bis 1995 dem SPD-Bezirksvorstand Mittelrhein an.\n\nAbgeordneter[Bearbeiten | Quelltext bearbeiten]\nVon 1975 bis 1998 war Großmann Ratsherr der Stadt Würselen.\nVon 1987 bis 2009 war er Mitglied des Deutschen Bundestages. Hier gehörte er von 1991 bis 1998 als wohnungspolitischer Sprecher dem Vorstand der SPD-Bundestagsfraktion an.\nAchim Großmann war stets als direkt gewählter Abgeordneter des Wahlkreises Kreis Aachen in den Bundestag eingezogen. Bei der Bundestagswahl 2005 erreichte er hier 46,0 % der Erststimmen. Bei der Bundestagswahl 2009 bewarb sich Großmann nicht mehr um ein Mandat.[2]\nÖffentliche Ämter[Bearbeiten | Quelltext bearbeiten]\nNach der Bundestagswahl 1998 wurde er am 27. Oktober 1998 als Parlamentarischer Staatssekretär beim Bundesminister für Verkehr, Bau- und Wohnungswesen in die von Bundeskanzler Gerhard Schröder geführte Bundesregierung berufen. Nach Bildung der Großen Koalition unter Bundeskanzlerin Angela Merkel wurde das Ministerium umbenannt in Bundesministerium für Verkehr, Bau und Stadtentwicklung. Nach der Bundestagswahl 2009 und dem folgenden Regierungswechsel schied Großmann im Oktober 2009 aus dem Amt. In den elf Jahren seiner Tätigkeit diente er unter fünf verschiedenen Bau- und Verkehrsministern.[2]Als Parlamentarischer Staatssekretär war Achim Großmann Aufsichtsratsmitglied der Deutschen Bahn und an den Vorbereitungen des Börsengangs beteiligt.\n\nKabinette[Bearbeiten | Quelltext bearbeiten]\nKabinett Schröder I – Kabinett Schröder II – Kabinett Merkel IVeröffentlichungen[Bearbeiten | Quelltext bearbeiten]\nWürselen – Geschichte(n) in alten Bildern. \nWürselener Ansichten. 1. Auflage. Buchhandlung Martin Schulz, Würselen 1987 (zusammen mit Josef Amberg). \nDie rothen Gesellen im schwarzen Westen. Die frühe Geschichte der sozialdemokratischen Bewegung in der Aachener Region. Hahne & Schloemer Verlag, Düren 2014, ISBN 978-3-942513-24-1. \nZigarren aus Würselen. Hahne & Schloemer Verlag, Düren 2015. Auszeichnungen[Bearbeiten | Quelltext bearbeiten]\n2009 – Weinritter – Auszeichnung der Stadt Oppenheim\n2015 – Baesweiler-Löwe\n2015 – Rheinlandtaler des Landschaftsverbandes Rheinland (LVR)Weblinks[Bearbeiten | Quelltext bearbeiten]\nLiteratur von und über Achim Großmann im Katalog der Deutschen Nationalbibliothek\nBiographie beim Deutschen BundestagEinzelnachweise[Bearbeiten | Quelltext bearbeiten]\n↑ Ehemaliger Staatssekretär Achim Großmann ist gestorben, in: Aachener Zeitung vom 18. April 2023\n\n↑ ab Verkehrsrundschau.de: Verkehrs-Staatssekretär Großmann tritt ab. 9. Juli 2008\n\nNormdaten (Person): GND: 1030018618 (lobid, OGND, AKS)  | LCCN: no2016104338  | VIAF: 295874549 | Wikipedia-Personensuche\nPersonendaten\nNAME\n\nGroßmann, Achim\nKURZBESCHREIBUNG\n\ndeutscher Politiker (SPD), MdB\nGEBURTSDATUM\n\n17. April 1947\nGEBURTSORT\n\nAachen\nSTERBEDATUM\n\n14. April 2023\nSTERBEORT\n\nWürselen\n
## 2                                                                                                                                                                                                                                                                                                                                                                                                                                                                                                                                                                                                                                                                                                                                                                                                                                                                                                                                                                                           Achim Kessler (2019)Achim Dieter Kessler (* 2. August 1964[1] in St. Georgen im Schwarzwald) ist ein deutscher Politiker (Die Linke). Von 2017 bis 2021 war er Mitglied des Deutschen Bundestags.\n\nInhaltsverzeichnis\n1 Leben\n2 Politik\n3 Schriften\n4 Weblinks\n5 Einzelnachweise\n\n\nLeben[Bearbeiten | Quelltext bearbeiten]\nNach dem Zivildienst in der Arbeiterwohlfahrt Villingen-Schwenningen 1986 folgte ein Studium der Neueren Deutschen Literatur und der Medienwissenschaften sowie der Wirtschafts- und Sozialgeschichte in Marburg. Während seines Studiums war Kessler Mitglied des Marxistischen Studentenbundes Spartakus und wurde in deren Bundesvorstand gewählt.[2] 1993 wurde er Mitglied im Vorstand des Lesben- und Schwulenverbandes Hessen und 1993–1994 Vorsitzender des AStA der Universität Marburg. 1998 wurde Kessler Vorstandsmitglied der AIDS-Hilfe Marburg e.V.\n2004 schloss er eine Promotion über Ernst Bloch an der Freien Universität Berlin ab.[3] Seit 2005 war er Wahlkreismitarbeiter des Bundestagsabgeordneten Wolfgang Gehrcke und seitdem Landespressesprecher der PDS von Hessen, später Die Linke. Hessen, wo er von 2010 bis 2014 stellvertretender Landesvorsitzender wurde. Von 2006 bis 2011 war Kessler ehrenamtlicher Stadtrat der Stadt Frankfurt am Main.[4]\nPolitik[Bearbeiten | Quelltext bearbeiten]\nNach Aktivitäten in der akademischen Selbstverwaltung und in der LGBT-Bewegung trat Achim Kessler 2001 der PDS bei. 2008 kandidierte er im Wahlkreis Main-Taunus I,[5] wo er 2,5 % der abgegebenen Stimmen erreichte. Zur Landtagswahl 2009 kandidierte er wieder als Direktkandidat im Wahlkreis 32.[6] Als Kandidat beschäftigte er sich vor allem mit sozialpolitischen Themen und dem Rentensystem.[7]\nBei der Bundestagswahl 2017 zog er über die Liste seiner Partei in den Deutschen Bundestag ein.\nIm 19. Deutschen Bundestag war Kessler Obmann des Ausschusses für Gesundheit und gehörte als stellvertretendes Mitglied dem Ausschuss für wirtschaftliche Zusammenarbeit und Entwicklung sowie dem Unterausschuss Globale Gesundheit an.[8]\nBei der Bundestagswahl 2021 gelang Kessler nicht der erneute Einzug als Abgeordneter in den Bundestag.[9] Im April 2022 berichtete Der Spiegel über Vorwürfe von Machtmissbrauch und sexuellen Übergriffen innerhalb des hessischen Landesverbands der Partei. Ein junger Mann erhob den Vorwurf, Kessler habe ihn im Oktober 2018 sexuell bedrängt. Kessler bestreitet diese Darstellung.[10] Am 11. Mai 2022 urteilte das Landgericht Wiesbaden, dass der Vorwurf gegen Achim Kessler nicht wiederholt werden darf, weil dem Gericht die vorgebrachten Belege nicht reichten.[11]\nSchriften[Bearbeiten | Quelltext bearbeiten]\n„Schafft die Einheit!“: die Figurenkonstellation in der „Ästhetik des Widerstands“ von Peter Weiss. In: Edition Philosophie und Sozialwissenschaften. Band 44. Argument, Berlin und Hamburg 1997, ISBN 3-88619-644-5, doi:10.1007/978-3-322-89585-1_13. \nErnst Blochs Ästhetik: Fragment, Montage, Metapher. Königshausen und Neumann, Würzburg 2006, ISBN 3-8260-3178-4. Weblinks[Bearbeiten | Quelltext bearbeiten]\nCommons: Achim Kessler – Sammlung von Bildern\nBiographie beim Deutschen Bundestag\nWebsite von Achim Kessler als Bundestagsabgeordneter (Stand 2021)\nAchim Kessler auf abgeordnetenwatch.deEinzelnachweise[Bearbeiten | Quelltext bearbeiten]\n↑ Biographie beim Deutschen Bundestag\n\n↑ Über mich – Achim KesslerbodybodybodyOriginal (nicht mehr online verfügbar) am 26. September 2017; abgerufen am 26. September 2017.  Info: Der Archivlink wurde automatisch eingesetzt und noch nicht geprüft. Bitte prüfe Original- und Archivlink gemäß Anleitung und entferne dann diesen Hinweis.@1@2Vorlage:Webachiv/IABot/www.achim-kessler.de \n\n↑ Achim Kessler: Ernst Blochs Ästhetik. Fragment, Montage, Metapher. Königshausen & Neumann. Würzburg 2006 ISBN 978-3-8260-3178-6\n\n↑ Achim Kessler bei Stadt Frankfurt (Memento des Originals vom 1. Dezember 2021 im Internet Archive)  Info: Der Archivlink wurde automatisch eingesetzt und noch nicht geprüft. Bitte prüfe Original- und Archivlink gemäß Anleitung und entferne dann diesen Hinweis.@1@2Vorlage:Webachiv/IABot/frankfurt.de\n\n↑ Abgeordnetenwatch\n\n↑ Frankfurter Rundschau: 6. Januar 2009\n\n↑ Osthessen-Zeitung vom 21. April 2017\n\n↑ Deutscher Bundestag - Abgeordnete. Abgerufen am 29. Juli 2020. \n\n↑ Gesundheitspolitiker im neuen Bundestag, 27. September 2021\n\n↑ Rafael Buschmann, Sophie Garbe, Timo Lehmann, Nicola Naber, Sara Wess: (S+) Vorwurf sexueller Übergriffe bei der Linken: »Entweder, wir brechen das jetzt, oder die Partei bricht«. In: Der Spiegel. 15. April 2022, ISSN 2195-1349 (spiegel.de [abgerufen am 29. November 2022]). \n\n↑ Prozess: Erstes Urteil zu Vorwürfen sexualisierter Gewalt bei der Linken. 11. Mai 2022, abgerufen am 21. Juni 2023. \n\nNormdaten (Person):   LCCN: no98103200  | VIAF: 14920038 | Wikipedia-Personensuche | Kein GND-Personendatensatz. Letzte Überprüfung: 26. Juni 2021. | Anmerkung: viaf:14920038 verknüpft mit GND 121599523 einen falschen Datensatz. Wegen falsch verknüpfter Literatur im März 2023 der DNB gemeldet\nPersonendaten\nNAME\n\nKessler, Achim\nALTERNATIVNAMEN\n\nKessler, Achim Dieter (vollständiger Name)\nKURZBESCHREIBUNG\n\ndeutscher Politiker (Die Linke), MdB\nGEBURTSDATUM\n\n2. August 1964\nGEBURTSORT\n\nSt. Georgen im Schwarzwald\n
## 3                                                                                                 Achim Post (2018)Achim Post (* 2. Mai 1959 in Rahden) ist ein deutscher Politiker (SPD). Er ist seit dem 26. August 2023 neben Sarah Philipp Co-Vorsitzender der SPD Nordrhein-Westfalen. \nEr war von 2002 bis 2012 stellvertretender Bundesgeschäftsführer der SPD und von 2012 bis 11. November 2023 Generalsekretär der Sozialdemokratischen Partei Europas (SPE). Seit 2013 ist er Mitglied des Deutschen Bundestages und seit 2017 ist er einer von acht stellvertretenden Vorsitzenden der SPD-Bundestagsfraktion.\n\nInhaltsverzeichnis\n1 Leben\n2 Partei\n3 Nebeneinkünfte\n4 Weblinks\n5 Einzelnachweise\n\n\nLeben[Bearbeiten | Quelltext bearbeiten]\nPost wuchs in Espelkamp auf, wo er 1978 sein Abitur am Söderblom-Gymnasium ablegte. Nach dem Zivildienst begann er 1980 ein Studium der Soziologie mit dem Schwerpunkt Öffentliche Verwaltung an der Universität Bielefeld, das er 1986 mit dem Diplom abschloss. Von 1986 bis 1990 arbeitete er als wissenschaftlicher Mitarbeiter für die Bundestagsabgeordneten Hans-Jürgen Wischnewski, Kurt Vogelsang und Dieter Heistermann und von 1990 bis 1999 war er als Referent, Büroleiter und Geschäftsführer der SPD-Gruppe im Europäischen Parlament tätig. Post ist evangelisch, verheiratet, hat zwei Kinder und lebt in Berlin. Sein Bruder ist der Filmregisseur Dietmar Post.\n\nPartei[Bearbeiten | Quelltext bearbeiten]\nEr bezeichnet Willy Brandt als prägend. Seinetwegen trat er 1976 in die SPD ein.[1] Von 1999 bis 2013 leitete er die Abteilung Internationale Politik beim SPD-Parteivorstand und von 2002 bis 2012 war er stellvertretender Bundesgeschäftsführer der SPD. Bei der Bundestagswahl 2009 kandidierte er als Direktkandidat im Bundestagswahlkreis Minden-Lübbecke I, unterlag aber dem CDU-Kandidaten Steffen Kampeter. Kampeter wurde Nachfolger von Lothar Ibrügger, der den Wahlkreis für die SPD bis 2009 vertreten hatte. Anfang Oktober 2012 wurde Post zum Generalsekretär der Sozialdemokratischen Partei Europas (SPE) gewählt.[2] Bei der Bundestagswahl 2013 kandidierte Post erneut als Direktkandidat im Wahlkreis Minden-Lübbecke I.[3] Er verlor den Wahlkreis gegen Steffen Kampeter, zog aber über die Landesliste in den Bundestag ein.[4] Am 23. September 2015 wurde Post zum Vorsitzenden der NRW-Landesgruppe in der SPD-Bundestagsfraktion gewählt.[5] Bei der Bundestagswahl 2017 gewann er im Wahlkreis Minden-Lübbecke I das Direktmandat.[6] Am 5. Dezember 2017 wurde Post zum stellvertretenden Vorsitzenden der SPD-Bundestagsfraktion für die Bereiche Europa, Haushalt und Finanzen gewählt.[7]Im 19. Deutschen Bundestag von 2017 bis 2021 war Post ordentliches Mitglied im Gemeinsamen Ausschuss. Zudem gehörte er als stellvertretendes Mitglied dem Finanzausschuss, dem Haushaltsausschuss, dem Vermittlungsausschuss, sowie dem Ausschuss für die Angelegenheiten der Europäischen Union an.[8]Am 26. August 2023 wurde er neben Sarah Philipp zum Co-Vorsitzenden der SPD Nordrhein-Westfalen gewählt.[9]\nNebeneinkünfte[Bearbeiten | Quelltext bearbeiten]\nAnfang 2014 tauchte Post auf der Liste der Bundestagsabgeordneten mit den höchsten Nebeneinkünften auf Platz elf auf. Nach Informationen von Abgeordnetenwatch.de verdiente Post zusätzlich zu seiner Tätigkeit im Parlament 75.000 Euro dazu und war damit der am besten verdienende Abgeordnete aus den Reihen der SPD-Fraktion. Grund war seine Tätigkeit als Generalsekretär der Sozialdemokratischen Partei Europas.[10] Seit Oktober 2014 übte Post die Tätigkeit unentgeltlich aus.[11]\nWeblinks[Bearbeiten | Quelltext bearbeiten]\nCommons: Achim Post – Sammlung von Bildern\nPersönliche Website\nBiographie beim Deutschen Bundestag\nAchim Post auf abgeordnetenwatch.deEinzelnachweise[Bearbeiten | Quelltext bearbeiten]\n↑ Lebenslauf Achim Post auf seiner Homepage\n\n↑ Mindener Tageblatt, abgerufen am 4. Oktober 2012.\n\n↑ Homepage Achim Post, abgerufen am 11. April 2012.\n\n↑ Bundeswahlleiter: Ergebnisse Wahlkreis 134 (Memento vom 27. September 2013 im Internet Archive)\n\n↑ Vorstand der NRW Landesgruppe der SPD, abgerufen am 24. September 2015.\n\n↑ Friederike Niemeyer und Arndt Hoppe: Achim Post gewinnt im Mühlenkreis. In: Westfalen-Blatt. 25. September 2017 (westfalen-blatt.de [abgerufen am 25. September 2017]). \n\n↑ Radio Westfalica: Post wird stellv. SPD Fraktionsvorsitzender, abgerufen am 5. Dezember 2017.\n\n↑ Deutscher Bundestag - Abgeordnete. Abgerufen am 11. November 2020. \n\n↑ SPD Nordrhein-Westfalen Pressemitteilung vom 26. August 2023: Sarah Philipp und Achim Post sind die neuen Vorsitzenden der NRWSPD, abgerufen am 26. August 2023\n\n↑ SPIEGEL ONLINE, Hamburg Germany: Ranking der Nebeneinkünfte: Gauweiler ist Top-Verdiener im Bundestag - SPIEGEL ONLINE - Politik. Abgerufen am 4. Juni 2017. \n\n↑ Achim Post, MdB - Für den Mühlenkreis Minden-Lübbecke in Berlin |   Bericht auf Abgeordnetenwatch.de. Abgerufen am 4. Juni 2017. \n\n\n\n\nAktuelle Landesvorsitzende der SPD\n\n\nBaden-Württemberg: Andreas Stoch |\nBayern: Florian von Brunn, Ronja Endres |\nBerlin: Franziska Giffey, Raed Saleh |\nBrandenburg: Dietmar Woidke |\nBremen: Reinhold Wetjen |\nHamburg: Melanie Leonhard, Nils Weiland |\nHessen: Nancy Faeser |\nMecklenburg-Vorpommern: Manuela Schwesig |\nNiedersachsen: Stephan Weil |\nNordrhein-Westfalen: Sarah Philipp, Achim Post |\nRheinland-Pfalz: Roger Lewentz |\nSaarland: Anke Rehlinger |\nSachsen: Kathrin Michel, Henning Homann |\nSachsen-Anhalt: Juliane Kleemann, Andreas Schmidt |\nSchleswig-Holstein: Serpil Midyatli |\nThüringen: Georg Maier\n\n\n\n\n\nLandesvorsitzende der SPD Nordrhein-Westfalen\n\n\nHeinz Kühn (1970–1973) |\nWerner Figgen (1973–1977) |\nJohannes Rau (1977–1998) |\nFranz Müntefering (1998–2001) |\nHarald Schartau (2001–2005) |\nJochen Dieckmann (2005–2007) |\nHannelore Kraft (2007–2017) |\nMichael Groschek (2017–2018) |\nSebastian Hartmann (2018–2021) |\nThomas Kutschaty (2021–2023) |\nMarc Herter (2023, komm.) |\nSarah Philipp und Achim Post (seit 2023)\n\n\n\n\n\nNormdaten (Person): GND: 1121690300 (lobid, OGND, AKS)    | VIAF: 209498404 | Wikipedia-Personensuche\nPersonendaten\nNAME\n\nPost, Achim\nKURZBESCHREIBUNG\n\ndeutscher Politiker (SPD), MdB\nGEBURTSDATUM\n\n2. Mai 1959\nGEBURTSORT\n\nRahden\n
## 4                                                                                                                                                                                                                                                                                                                                                                                                                                                                                                                                                                                                                                                                                                                                                                                                                                                                                                                                                                                                                                                                                                                                                                                                                                                                                                                                                                                                                                                                                                                                                                                                                                                                                                                                                                                                                                                                                                                                                                                                                                                                                                                                                                                                                                                                                                                                                                                                                                                                                                                                                                                                                                                                                                                                                     Adelheid D. Tröscher (* 16. Februar 1939 in Berlin) ist eine deutsche Pädagogin und Politikerin (SPD). Sie war von 1994 bis 2002 Mitglied des Deutschen Bundestages.\n\nInhaltsverzeichnis\n1 Leben\n2 Politik\n3 Mitgliedschaften\n4 Literatur\n5 Weblinks\n6 Einzelnachweise\n\n\nLeben[Bearbeiten | Quelltext bearbeiten]\n\n\n\nDieser Artikel oder nachfolgende Abschnitt ist nicht hinreichend mit Belegen (beispielsweise Einzelnachweisen) ausgestattet. Angaben ohne ausreichenden Beleg könnten demnächst entfernt werden. Bitte hilf Wikipedia, indem du die Angaben recherchierst und gute Belege einfügst.\n\nTröscher besuchte das Gymnasium und die High School in Ithaca (New York). 1959 legte sie ihr Abitur ab. Anschließend studierte sie Pädagogik, Geschichte und Anglistik in Marburg, Jugenheim an der Bergstraße und Frankfurt am Main und legte die Erste und Zweite Staatsprüfung für das Lehramt an Haupt- und Realschulen ab. 1962 wurde sie in den Schuldienst aufgenommen und von 1975 bis 1976 war sie in der Schulabteilung des Regierungspräsidiums Darmstadt tätig. Von 1986 bis 1992 war sie Leiterin der Paul-Hindemith-Gesamtschule in Frankfurt am Main. \nNach selbstbestimmtem Ausscheiden aus der Bundestagsfraktion engagiert sie sich ehrenamtlich in der SPD, besonders im Forum eine Welt, das sich mit entwicklungspolitischen Fragestellungen befasst. Außerdem leitet sie die Regionalgruppe Rhein-Main von Transparency International. Weiter engagiert sie sich als Vorsitzende in der Stiftung Citoyen, die das bürgerschaftliche Engagement unterstützt und sich besonders um musisch-ästhetische Bildung von Kindern und Jugendlichen kümmert.\nSie ist die Tochter von Tassilo Tröscher.\n\nPolitik[Bearbeiten | Quelltext bearbeiten]\nTröscher trat 1966 in die SPD ein. Von 1972 bis 1977, von 1981 bis 1989 und im Jahre 1991 war sie Stadtverordnete der Stadt Frankfurt am Main, von 1992 bis 1993 hauptamtliche Kreisbeigeordnete des Kreises Offenbach.\nVon November 1994 bis Oktober 2002 war Tröscher für zwei Wahlperioden Mitglied des Deutschen Bundestages. Bei der Bundestagswahl 1994 zog sie über die Landesliste der SPD ins Parlament ein, bei der Bundestagswahl 1998 gewann sie das Direktmandat im Wahlkreis 144 (Odenwald). Von 1994 bis 2002 war sie Mitglied des Ausschusses für wirtschaftliche Zusammenarbeit und Entwicklung und von 1996 bis 2002 entwicklungspolitische Sprecherin der SPD-Bundestagsfraktion.\n\nMitgliedschaften[Bearbeiten | Quelltext bearbeiten]\nTröscher ist Mitglied der GEW, der AWO, im Kinderschutzbund, in Pro Familia und von medico international. 1999 wurde sie Mitglied des Kuratoriums und des Verwaltungsausschusses der Deutschen Stiftung für internationale Entwicklung (DSE) und Mitglied der Deutschen Welthungerhilfe.[1]\nLiteratur[Bearbeiten | Quelltext bearbeiten]\nRudolf Vierhaus, Ludolf Herbst (Hrsg.), Bruno Jahn (Mitarb.): Biographisches Handbuch der Mitglieder des Deutschen Bundestages. 1949–2002. Bd. 2: N–Z. Anhang. K. G. Saur, München 2002, ISBN 3-598-23782-0, S. 885.Weblinks[Bearbeiten | Quelltext bearbeiten]\nBiographie beim Deutschen BundestagEinzelnachweise[Bearbeiten | Quelltext bearbeiten]\n↑ Adelheid Tröscher. Abgeordnete der 14. Legislaturperiode. SPD-Bundestagsfraktion, abgerufen am 16. März 2017. \n\nNormdaten (Person): GND: 173187528 (lobid, OGND, AKS)    | VIAF: 241645083 | Wikipedia-Personensuche | Letzte Überprüfung: 13. Oktober 2018. GND-Namenseintrag: 118160591 (AKS)\nPersonendaten\nNAME\n\nTröscher, Adelheid\nALTERNATIVNAMEN\n\nTröscher, Adelheid D.\nKURZBESCHREIBUNG\n\ndeutsche Politikerin (SPD), MdB\nGEBURTSDATUM\n\n16. Februar 1939\nGEBURTSORT\n\nBerlin\n
## 5 Adis Ahmetovic, 2021Adis Ahmetovic (Aussprache [axměːtoʋitɕ]; bosnisch Ahmetović; * 27. Juli 1993 in Hannover[1]) ist ein deutscher Politiker (SPD). Seit 2021 ist er Mitglied des Deutschen Bundestages.\n\nInhaltsverzeichnis\n1 Leben und Beruf\n2 Politische Tätigkeiten\n3 Politische Positionen\n4 Mitgliedschaften\n5 Privates\n6 Weblinks\n7 Einzelnachweise\n\n\nLeben und Beruf[Bearbeiten | Quelltext bearbeiten]\nDie Eltern Ahmetovics stammen aus Bosnien und Herzegowina. Wegen der Jugoslawienkriege flohen sie 1992 nach Hannover, dort wurde Adis Ahmetovic 1993 geboren. Sein Vater Fuad, der zuvor in Kotor Varoš als Verwaltungsjurist gearbeitet hatte, arbeitete in Deutschland auf dem Bau und später als Lagerist, seine Mutter Edina als Reinigungskraft.[1][2] 1996 sollte die Familie abgeschoben werden, dies verhinderte der Rechtsanwalt der Familie, Matthias Miersch, dessen Fraktionskollege Ahmetovic später wurde.[3]Ahmetovic besuchte ab 2004 die Herschelschule Hannover und schloss diese 2011 mit dem Abitur ab. An der Leibniz Universität Hannover studierte er Politik-Wirtschaft und Germanistik auf Lehramt. Während seines Studiums von 2011 bis 2016 war er Stipendiat der Friedrich-Ebert-Stiftung. 2015 erhielt er seinen Bachelor of Arts. 2019 schloss er sein Studium mit dem Master of Education ab.[4][5][6]Während des Studiums arbeitete Ahmetovic 2013 als wissenschaftlicher Mitarbeiter der Bundestagsabgeordneten Kerstin Tack. 2015 wurde Ahmetovic Leiter des Wahlkreisbüros des damaligen SPD-Mitglieds und Landtagsabgeordneten Mustafa Erkan, zuvor war Ahmetovic dort als wissenschaftlicher Mitarbeiter im Büro tätig.[7] Von 2016 bis 2020 arbeitete er zunächst als Büroleiter und später als persönlicher Referent des Niedersächsischen Ministerpräsidenten und Vorsitzenden der SPD Niedersachsen Stephan Weil. Von 2020 bis zu seiner Wahl in den Deutschen Bundestag gehörte er als Referent „Regierungsplanung und Grundsatzfragen“ der Niedersächsischen Staatskanzlei an.[6][8]\nPolitische Tätigkeiten[Bearbeiten | Quelltext bearbeiten]\nAdis Ahmetovic trat 2008 in die Sozialdemokratische Partei Deutschlands ein. Er war 2014 bis 2018 Vorsitzender der Jusos in der Region Hannover und ist seit 2020 einer der beiden Vorsitzenden der SPD Hannover.\nBei den Kommunalwahlen in Niedersachsen 2016 wurde Ahmetovic in den Bezirksrat Bothfeld-Vahrenheide gewählt.[9] Bei den Kommunalwahlen in Niedersachsen 2021 trat er nicht nochmal an.\nBei der Bundestagswahl 2021 errang Ahmetovic das Direktmandat für den Bundestagswahlkreis Stadt Hannover I mit 34,9 % der Erststimmen.[10][11] Er ist ordentliches Mitglied des Auswärtigen Ausschusses und stellvertretendes Mitglied des Verteidigungsausschusses.\n\nPolitische Positionen[Bearbeiten | Quelltext bearbeiten]\nAdis Ahmetovic mit Boris Pistorius 2021Ahmetovic strebt höhere Investitionen in Bildung, bessere Pflege und günstigeren Wohnraum an. Den Klimawandel möchte er u. a. mit der Einführung eines 365-Euro-Ticket bekämpfen. Zudem will er Hannover als Arbeitsstandort stärken.[12] Als Außenpolitiker setzt er sich unter anderem für einen schnellen EU-Beitrittsprozesses der gesamten Westbalkan-Region ein.[13]\nMitgliedschaften[Bearbeiten | Quelltext bearbeiten]\nSeit 2017 ist er Vorsitzender des Integrationsbeirates Bothfeld-Vahrenheide. Seit November 2020 ist Ahmetovic Vizepräsident des Deutschen Roten Kreuzes Region Hannover.[6][14] Zudem ist er Mitglied der Gewerkschaft Erziehung und Wissenschaft, der Arbeiterwohlfahrt, des Arbeiter-Samariter-Bund Deutschland und des Vereins Mach meinen Kumpel nicht an!.\n\nPrivates[Bearbeiten | Quelltext bearbeiten]\nIm Alter von 22 Jahren erlangte Ahmetovic die deutsche Staatsangehörigkeit und gab seinen bosnischen Pass ab.[1][13] Er beherrscht neben der deutschen auch die bosnische Sprache.[13]Ahmetovic ist nach eigenen Angaben ledig,[6]konfessionslos,[15] hat einen älteren Bruder und wohnt in Hannover-Bothfeld.[16]\nWeblinks[Bearbeiten | Quelltext bearbeiten]\nCommons: Adis Ahmetovic – Sammlung von Bildern, Videos und Audiodateien\nOffizielle Website von Adis Ahmetovic\nBiographie beim Deutschen Bundestag\nAdis Ahmetovic auf abgeordnetenwatch.deEinzelnachweise[Bearbeiten | Quelltext bearbeiten]\n↑ abc Zoran Pantic:  Es war eine gute Entscheidung, Neue Presse, 10. September 2015. Abgerufen am 17. September 2021. \n\n↑  Vom Flüchtling zum Politiker, pangea, 7. September 2016. Abgerufen am 17. September 2021. \n\n↑ Für die SPD Hannover im Bundestag: Was Ahmetovic und Miersch verbindet. Abgerufen am 26. Oktober 2021. \n\n↑  Stipendiatenporträts, Stipendiumplus. Abgerufen am 17. September 2021. \n\n↑  Absolventinnen- und Absolventenfeier der Philosophischen Fakultät und der Leibniz School of Education, Leibniz Universität Hannover. Abgerufen am 17. September 2021. \n\n↑ abcd Adis Ahmetovic: Lebenslauf. Abgerufen am 17. September 2021. \n\n↑ Kleiner Start in ein großes Jahr? Archiviert vom bodybodybodyOriginal (nicht mehr online verfügbar) am 4. Oktober 2021; abgerufen am 4. Oktober 2021. \n\n↑  Ahmetovic wechselt, Hansmann folgt, Rundblick – Politikjournal für Niedersachsen, 12. Februar 2020. Abgerufen am 17. September 2021. \n\n↑ Wahlergebnisse 2016. SPD-Ortsverein Vahrenheide / Sahlkamp, abgerufen am 17. September 2021. \n\n↑  SPD holt alle vier Direktmandate in der Region: Newcomerin Schamber gewinnt gegen Hoppenstedt, Hannoversche Allgemeine Zeitung, 26. September 2021 \n\n↑ Ergebnisse Stadt Hannover I - Der Bundeswahlleiter. Abgerufen am 3. April 2022. \n\n↑ Adis Ahmetovic: Dafür stehe ich - in 60 Sekunden. Abgerufen am 17. September 2021. \n\n↑ abc Laura Meyer:  Spiegel-Interview. Abgerufen am 24. Juli 2023. \n\n↑ Das Präsidium des DRK-Region Hannover e.V. Deutsches Rotes Kreuz, abgerufen am 17. September 2021. \n\n↑  Über den Krieg und den Papst, Cityglow. Abgerufen am 17. September 2021. \n\n↑ Adis Ahmetovic:  Unser Leben in Hannovers Norden. Abgerufen am 17. September 2021. \n\n\nNormdaten (Person):      Wikipedia-Personensuche | Kein GND-Personendatensatz. Letzte Überprüfung: 30. September 2021.\nPersonendaten\nNAME\n\nAhmetovic, Adis\nALTERNATIVNAMEN\n\nAhmetović, Adis (bosnisch)\nKURZBESCHREIBUNG\n\ndeutscher Politiker (SPD) und Mitglied des Bundestags\nGEBURTSDATUM\n\n27. Juli 1993\nGEBURTSORT\n\nHannover\n
## 6                                                                                                                                                                                                                                                                                                                                                                                                                                                                                                                                                                                                                                                                                                                                                                                                                                                                                                                                                                                                                                                                                                                                                                                                                                                                                                                                                                                                                                                                                                                                                                                                                                                                                                                                                                                                                                                                                                                                                                                                                                                                                                                                                                                                                                                                                                                                                                                                                                                                                                                                                                                                                                                                                                                                                                                                                                                                                                                                                                                                                                                                                                                                                                                                                                                                                                                                                                                                                                                                                                                                                                                                                                                          Adolf „Adi“ Ostertag (* 22. Juli 1939 in Penzendorf) ist ein deutscher Politiker (SPD) und ehemaliger Abgeordneter des Deutschen Bundestages.\n\nInhaltsverzeichnis\n1 Leben\n2 Politik\n3 Veröffentlichungen (Auswahl)\n4 Weblinks\n5 Anmerkungen\n\n\nLeben[Bearbeiten | Quelltext bearbeiten]\nOstertag wurde 1939 in Mittelfranken geboren. Nach dem Besuch der Volksschule begann er 1953 eine Lehre als Werkzeugmacher. Anschließend arbeitete er auch in diesem Beruf, bis er 1964 an der Hochschule für Wirtschaft und Politik in Hamburg studierte. Er verweigerte den Kriegsdienst und arbeitete ab 1967 als Gewerkschaftssekretär bei der IG Metall.\n\nPolitik[Bearbeiten | Quelltext bearbeiten]\nOstertag trat mit dem Beginn seiner Ausbildung der IG Metall bei, wo er sich auch engagierte. Er war als Jugendvertreter und im Betriebsrat aktiv und wurde sogar Betriebsratsvorsitzender. Im Jahr 1968 trat er der SPD bei und wurde 1970 Vorstandsmitglied der Jungsozialisten im südlichen Hessen, was er bis 1971 blieb. Ein Jahr später wurde er Vorsitzender der Arbeitsgemeinschaft für Arbeit in Sprockhövel, ab 1978 war er in wechselnden Parteifunktionen tätig. Er blieb jedoch in Sprockhövel und wurde 1971 pädagogischer Leiter des IG Metall Bildungszentrums, was er bis zu seinem Einzug in den Deutschen Bundestag blieb. Von 1990 bis 2002 war er Bundestagsabgeordneter für die SPD und erlangte bei den Wahlen jeweils ein Direktmandat im Wahlkreis Ennepe-Ruhr-Kreis I. Interessenschwerpunkt im Bundestag war die Sozialpolitik. Im Jahr 1996 wurde er Vorsitzender der SPD-Landesgruppe Nordrhein-Westfalen im Bundestag, zwei Jahre später wurde er Mitglied des SPD-Fraktionsvorstandes und sozialpolitischer Sprecher der SPD-Fraktion.\n\nVeröffentlichungen (Auswahl)[Bearbeiten | Quelltext bearbeiten]\n(Hrsg.): Arbeitsdirektoren berichten aus der Praxis, Bund-Verlag, 1981.\nmit Otto König, Hartmut Schulz: „Unser Beispiel könnte ja Schule machen!“. Das „Hattinger Modell“ Existenzkampf an der Ruhr. Bund-Verlag, 1985, ISBN 3-7663-0924-2.[1]Weblinks[Bearbeiten | Quelltext bearbeiten]\nBiographie beim Deutschen BundestagAnmerkungen[Bearbeiten | Quelltext bearbeiten]\n↑ Hhierzu erfolgten allein bis zum 18. Oktober 1985 29 Besprechungen, darunter in der Frankfurter Rundschau (11. und 29. März 1985), im Rundfunk (WDR 1) vom 14. März, im WDR Fernsehen (18. März), in der Tageszeitung vom 5. August 1985.\n\nNormdaten (Person): GND: 107960281X (lobid, OGND, AKS)  | LCCN: n81069559  | VIAF: 59661983 | Wikipedia-Personensuche\nPersonendaten\nNAME\n\nOstertag, Adolf\nALTERNATIVNAMEN\n\nOstertag, Adi (Spitzname)\nKURZBESCHREIBUNG\n\ndeutscher Politiker (SPD) und ehemaliger Abgeordneter des Deutschen Bundestages\nGEBURTSDATUM\n\n22. Juli 1939\nGEBURTSORT\n\nPenzendorf\n
\end{verbatim}

\begin{Shaded}
\begin{Highlighting}[]
\FunctionTok{summary}\NormalTok{(all\_countries)}
\end{Highlighting}
\end{Shaded}

\begin{verbatim}
##    country             pageid           wikidataid         wikititle        
##  Length:14064       Length:14064       Length:14064       Length:14064      
##  Class :character   Class :character   Class :character   Class :character  
##  Mode  :character   Mode  :character   Mode  :character   Mode  :character  
##      name               sex             ethnicity           religion        
##  Length:14064       Length:14064       Length:14064       Length:14064      
##  Class :character   Class :character   Class :character   Class :character  
##  Mode  :character   Mode  :character   Mode  :character   Mode  :character  
##     birth            death          birthplace         deathplace       
##  Length:14064       Mode:logical   Length:14064       Length:14064      
##  Class :character   NA's:14064     Class :character   Class :character  
##  Mode  :character                  Mode  :character   Mode  :character  
##   plain_text       
##  Length:14064      
##  Class :character  
##  Mode  :character
\end{verbatim}

\begin{Shaded}
\begin{Highlighting}[]
\FunctionTok{str}\NormalTok{(all\_countries)}
\end{Highlighting}
\end{Shaded}

\begin{verbatim}
## 'data.frame':    14064 obs. of  13 variables:
##  $ country   : chr  "DEU" "DEU" "DEU" "DEU" ...
##  $ pageid    : chr  "174000" "9980355" "5166669" "261258" ...
##  $ wikidataid: chr  "Q340387" "Q39678866" "Q340448" "Q354647" ...
##  $ wikititle : chr  "Achim_Großmann" "Achim_Kessler" "Achim_Post" "Adelheid_Tröscher" ...
##  $ name      : chr  "Achim Großmann" "Achim Kessler" "Achim Post" "Adelheid D. Tröscher" ...
##  $ sex       : chr  "male" "male" "male" "female" ...
##  $ ethnicity : chr  NA "white" "white" NA ...
##  $ religion  : chr  "catholicism" NA "protestantism lutheran" NA ...
##  $ birth     : chr  "1947-04-17" "1964-08-02" "1959-05-02" "1939-02-16" ...
##  $ death     : logi  NA NA NA NA NA NA ...
##  $ birthplace: chr  "50.77621,6.08379" "48.12472,8.33083" "52.41667,8.61667" "52.51667,13.38333" ...
##  $ deathplace: chr  NA NA NA NA ...
##  $ plain_text: chr  "Achim Großmann (* 17. April 1947 in Aachen; † 14. April 2023 in Würselen[1]) war ein deutscher Politiker (SPD)."| __truncated__ "Achim Kessler (2019)Achim Dieter Kessler (* 2. August 1964[1] in St. Georgen im Schwarzwald) ist ein deutscher "| __truncated__ "Achim Post (2018)Achim Post (* 2. Mai 1959 in Rahden) ist ein deutscher Politiker (SPD). Er ist seit dem 26. Au"| __truncated__ "Adelheid D. Tröscher (* 16. Februar 1939 in Berlin) ist eine deutsche Pädagogin und Politikerin (SPD). Sie war "| __truncated__ ...
\end{verbatim}

\begin{Shaded}
\begin{Highlighting}[]
\CommentTok{\#Plot the number of female/male politicians per country}
\FunctionTok{ggplot}\NormalTok{(all\_countries, }\FunctionTok{aes}\NormalTok{(}\AttributeTok{x =}\NormalTok{ sex, }\AttributeTok{fill =}\NormalTok{ sex)) }\SpecialCharTok{+}
  \FunctionTok{geom\_bar}\NormalTok{() }\SpecialCharTok{+}
  \FunctionTok{facet\_wrap}\NormalTok{(}\SpecialCharTok{\textasciitilde{}}\NormalTok{country, }\AttributeTok{scales =} \StringTok{"free\_y"}\NormalTok{) }\SpecialCharTok{+}
  \FunctionTok{labs}\NormalTok{(}\AttributeTok{title =} \StringTok{"Number of male/female politicians per country"}\NormalTok{) }\SpecialCharTok{+}
  \FunctionTok{xlab}\NormalTok{(}\StringTok{"sex"}\NormalTok{) }\SpecialCharTok{+}
  \FunctionTok{ylab}\NormalTok{(}\StringTok{"number"}\NormalTok{) }\SpecialCharTok{+}
  \FunctionTok{scale\_fill\_manual}\NormalTok{(}\AttributeTok{values =} \FunctionTok{c}\NormalTok{(}\StringTok{"male"} \OtherTok{=} \StringTok{"blue"}\NormalTok{, }\StringTok{"female"} \OtherTok{=} \StringTok{"pink"}\NormalTok{)) }\SpecialCharTok{+}
  \FunctionTok{theme\_minimal}\NormalTok{() }\SpecialCharTok{+}
  \FunctionTok{theme}\NormalTok{(}\AttributeTok{legend.title =} \FunctionTok{element\_blank}\NormalTok{())}
\end{Highlighting}
\end{Shaded}

\includegraphics{Data_Report_files/figure-latex/unnamed-chunk-9-1.pdf}
We can see, that female politicians, as it can be expected, are
underrepresented in all countries. Still, it leaves us with a decent
number of female politicians to be compared to the male politicians.

Next, we want to have a look at the average monthly number of page views
(traffic). This will be used to match female and male politicians in
order to make them more comparable. Using this varibale as matching
variable is explained by the hypothesis that fame represents the primary
confounder among the results, when it comes to the length of texts and
the number of edits. As a proxy variable for fame this measure ensures
that the analysis only examines comparable men and women.

\begin{Shaded}
\begin{Highlighting}[]
\NormalTok{deu\_traffic }\OtherTok{\textless{}{-}} \FunctionTok{get\_traffic}\NormalTok{(}\AttributeTok{legislature =} \StringTok{"deu"}\NormalTok{)}
\NormalTok{deu\_average\_traffic }\OtherTok{\textless{}{-}} \FunctionTok{traffic\_metrics}\NormalTok{(deu\_traffic)}
\NormalTok{deu }\OtherTok{\textless{}{-}} \FunctionTok{left\_join}\NormalTok{(deu, }\FunctionTok{select}\NormalTok{(deu\_average\_traffic, pageid, average\_traffic), }\AttributeTok{by =} \StringTok{"pageid"}\NormalTok{)}

\NormalTok{fra\_traffic }\OtherTok{\textless{}{-}} \FunctionTok{get\_traffic}\NormalTok{(}\AttributeTok{legislature =} \StringTok{"fra"}\NormalTok{)}
\NormalTok{fra\_average\_traffic }\OtherTok{\textless{}{-}} \FunctionTok{traffic\_metrics}\NormalTok{(fra\_traffic)}
\NormalTok{fra }\OtherTok{\textless{}{-}} \FunctionTok{left\_join}\NormalTok{(fra, }\FunctionTok{select}\NormalTok{(fra\_average\_traffic, pageid, average\_traffic), }\AttributeTok{by =} \StringTok{"pageid"}\NormalTok{)}

\CommentTok{\#error}
\NormalTok{gbr\_traffic }\OtherTok{\textless{}{-}} \FunctionTok{get\_traffic}\NormalTok{(}\AttributeTok{legislature =} \StringTok{"gbr"}\NormalTok{)}
\NormalTok{gbr\_average\_traffic }\OtherTok{\textless{}{-}} \FunctionTok{traffic\_metrics}\NormalTok{(gbr\_traffic)}
\NormalTok{gbr}\SpecialCharTok{$}\NormalTok{pageid }\OtherTok{\textless{}{-}} \FunctionTok{as.character}\NormalTok{(gbr}\SpecialCharTok{$}\NormalTok{pageid)}
\NormalTok{gbr\_average\_traffic}\SpecialCharTok{$}\NormalTok{pageid }\OtherTok{\textless{}{-}} \FunctionTok{as.character}\NormalTok{(gbr\_average\_traffic}\SpecialCharTok{$}\NormalTok{pageid)}
\NormalTok{gbr }\OtherTok{\textless{}{-}} \FunctionTok{left\_join}\NormalTok{(gbr, }\FunctionTok{select}\NormalTok{(gbr\_average\_traffic, pageid, average\_traffic), }\AttributeTok{by =} \StringTok{"pageid"}\NormalTok{)}
\NormalTok{gbr}\SpecialCharTok{$}\NormalTok{pageid }\OtherTok{\textless{}{-}} \FunctionTok{as.numeric}\NormalTok{(gbr}\SpecialCharTok{$}\NormalTok{pageid)}

\NormalTok{can\_traffic }\OtherTok{\textless{}{-}} \FunctionTok{get\_traffic}\NormalTok{(}\AttributeTok{legislature =} \StringTok{"can"}\NormalTok{)}
\NormalTok{can\_average\_traffic }\OtherTok{\textless{}{-}} \FunctionTok{traffic\_metrics}\NormalTok{(can\_traffic)}
\NormalTok{can }\OtherTok{\textless{}{-}} \FunctionTok{left\_join}\NormalTok{(can, }\FunctionTok{select}\NormalTok{(can\_average\_traffic, pageid, average\_traffic), }\AttributeTok{by =} \StringTok{"pageid"}\NormalTok{)}

\NormalTok{aut\_traffic }\OtherTok{\textless{}{-}} \FunctionTok{get\_traffic}\NormalTok{(}\AttributeTok{legislature =} \StringTok{"aut"}\NormalTok{)}
\NormalTok{aut\_average\_traffic }\OtherTok{\textless{}{-}} \FunctionTok{traffic\_metrics}\NormalTok{(aut\_traffic)}
\NormalTok{aut }\OtherTok{\textless{}{-}} \FunctionTok{left\_join}\NormalTok{(aut, }\FunctionTok{select}\NormalTok{(aut\_average\_traffic, pageid, average\_traffic), }\AttributeTok{by =} \StringTok{"pageid"}\NormalTok{)}


\CommentTok{\# introduces NAs exclusively, need to look into that}
\NormalTok{esp\_traffic }\OtherTok{\textless{}{-}} \FunctionTok{get\_traffic}\NormalTok{(}\AttributeTok{legislature =} \StringTok{"esp"}\NormalTok{)}
\NormalTok{esp\_average\_traffic }\OtherTok{\textless{}{-}} \FunctionTok{traffic\_metrics}\NormalTok{(esp\_traffic)}
\end{Highlighting}
\end{Shaded}

\begin{verbatim}
## Warning in traffic_metrics(esp_traffic): NAs introduced by coercion
\end{verbatim}

\begin{Shaded}
\begin{Highlighting}[]
\NormalTok{esp}\SpecialCharTok{$}\NormalTok{pageid }\OtherTok{\textless{}{-}} \FunctionTok{as.character}\NormalTok{(esp}\SpecialCharTok{$}\NormalTok{pageid)}
\NormalTok{esp\_average\_traffic}\SpecialCharTok{$}\NormalTok{pageid }\OtherTok{\textless{}{-}} \FunctionTok{as.character}\NormalTok{(esp\_average\_traffic}\SpecialCharTok{$}\NormalTok{pageid)}
\NormalTok{esp }\OtherTok{\textless{}{-}} \FunctionTok{left\_join}\NormalTok{(esp, }\FunctionTok{select}\NormalTok{(esp\_average\_traffic, pageid, average\_traffic), }\AttributeTok{by =} \StringTok{"pageid"}\NormalTok{)}
\NormalTok{esp}\SpecialCharTok{$}\NormalTok{pageid }\OtherTok{\textless{}{-}} \FunctionTok{as.numeric}\NormalTok{(esp}\SpecialCharTok{$}\NormalTok{pageid)}
\end{Highlighting}
\end{Shaded}

\begin{verbatim}
## Warning: NAs introduced by coercion
\end{verbatim}

\begin{Shaded}
\begin{Highlighting}[]
\NormalTok{cze\_traffic }\OtherTok{\textless{}{-}} \FunctionTok{get\_traffic}\NormalTok{(}\AttributeTok{legislature =} \StringTok{"cze"}\NormalTok{)}
\NormalTok{cze\_average\_traffic }\OtherTok{\textless{}{-}} \FunctionTok{traffic\_metrics}\NormalTok{(cze\_traffic)}
\NormalTok{cze }\OtherTok{\textless{}{-}} \FunctionTok{left\_join}\NormalTok{(cze, }\FunctionTok{select}\NormalTok{(cze\_average\_traffic, pageid, average\_traffic), }\AttributeTok{by =} \StringTok{"pageid"}\NormalTok{)}

\NormalTok{sco\_traffic }\OtherTok{\textless{}{-}} \FunctionTok{get\_traffic}\NormalTok{(}\AttributeTok{legislature =} \StringTok{"sco"}\NormalTok{)}
\NormalTok{sco\_average\_traffic }\OtherTok{\textless{}{-}} \FunctionTok{traffic\_metrics}\NormalTok{(sco\_traffic)}
\NormalTok{sco }\OtherTok{\textless{}{-}} \FunctionTok{left\_join}\NormalTok{(sco, }\FunctionTok{select}\NormalTok{(sco\_average\_traffic, pageid, average\_traffic), }\AttributeTok{by =} \StringTok{"pageid"}\NormalTok{)}

\NormalTok{irl\_traffic }\OtherTok{\textless{}{-}} \FunctionTok{get\_traffic}\NormalTok{(}\AttributeTok{legislature =} \StringTok{"irl"}\NormalTok{)}
\NormalTok{irl\_average\_traffic }\OtherTok{\textless{}{-}} \FunctionTok{traffic\_metrics}\NormalTok{(irl\_traffic)}
\NormalTok{irl }\OtherTok{\textless{}{-}} \FunctionTok{left\_join}\NormalTok{(irl, }\FunctionTok{select}\NormalTok{(irl\_average\_traffic, pageid, average\_traffic), }\AttributeTok{by =} \StringTok{"pageid"}\NormalTok{)}

\NormalTok{usa\_house\_traffic }\OtherTok{\textless{}{-}} \FunctionTok{get\_traffic}\NormalTok{(}\AttributeTok{legislature =} \StringTok{"usa\_house"}\NormalTok{)}
\NormalTok{usa\_senate\_traffic }\OtherTok{\textless{}{-}} \FunctionTok{get\_traffic}\NormalTok{(}\AttributeTok{legislature =} \StringTok{"usa\_senate"}\NormalTok{)}

\NormalTok{usa\_traffic }\OtherTok{\textless{}{-}} \FunctionTok{bind\_rows}\NormalTok{(usa\_house\_traffic, usa\_senate\_traffic)}
\NormalTok{usa\_average\_traffic }\OtherTok{\textless{}{-}} \FunctionTok{traffic\_metrics}\NormalTok{(usa\_traffic)}
\NormalTok{usa }\OtherTok{\textless{}{-}} \FunctionTok{left\_join}\NormalTok{(usa, }\FunctionTok{select}\NormalTok{(usa\_average\_traffic, pageid, average\_traffic), }\AttributeTok{by =} \StringTok{"pageid"}\NormalTok{)}
\end{Highlighting}
\end{Shaded}

Let's get some insights on the average traffic variable for all the
countries by looking at the top pages and creating a boxplot per country
and per sex

\begin{Shaded}
\begin{Highlighting}[]
\NormalTok{all\_countries\_traffic }\OtherTok{\textless{}{-}} \FunctionTok{rbind}\NormalTok{(deu, cze, fra, usa, sco, irl, can, aut, esp, gbr) }\CommentTok{\#without esp, gbr }
\NormalTok{all\_countries\_traffic }\OtherTok{\textless{}{-}}\NormalTok{ all\_countries\_traffic}\SpecialCharTok{\%\textgreater{}\%}
  \FunctionTok{filter}\NormalTok{(}\SpecialCharTok{!}\FunctionTok{is.na}\NormalTok{(sex))}


\NormalTok{all\_countries\_traffic }\SpecialCharTok{\%\textgreater{}\%}
  \FunctionTok{group\_by}\NormalTok{(country) }\SpecialCharTok{\%\textgreater{}\%}
  \FunctionTok{arrange}\NormalTok{(}\FunctionTok{desc}\NormalTok{(average\_traffic)) }\SpecialCharTok{\%\textgreater{}\%}
  \FunctionTok{slice\_head}\NormalTok{(}\AttributeTok{n =} \DecValTok{3}\NormalTok{)}
\end{Highlighting}
\end{Shaded}

\begin{verbatim}
## # A tibble: 33 x 14
## # Groups:   country [11]
##    country   pageid wikidataid wikititle    name  sex   ethnicity religion birth
##    <chr>      <dbl> <chr>      <chr>        <chr> <chr> <chr>     <chr>    <chr>
##  1 AUT      4554427 Q2262885   Sebastian_K~ Kurz~ male  white     catholi~ 1986~
##  2 AUT      1936075 Q44733     Heinz-Chris~ Stra~ male  white     <NA>     1969~
##  3 AUT       169973 Q78869     Alexander_V~ Van ~ male  white     catholi~ 1944~
##  4 CAN       451733 Q3099714   Justin_Trud~ Just~ male  white     catholi~ 1971~
##  5 CAN       241547 Q206       Stephen_Har~ Step~ male  white     protest~ 1959~
##  6 CAN     33331329 Q6122681   Jagmeet_Sin~ Jagm~ male  <NA>      <NA>     1979~
##  7 CZE       336591 Q10819807  Andrej_Babiš Andr~ male  white     <NA>     1954~
##  8 CZE        13605 Q29032     Miloš_Zeman  Milo~ male  white     atheism  1944~
##  9 CZE        86549 Q58195     Karel_Schwa~ Kare~ male  white     catholi~ 1937~
## 10 DEU          145 Q567       Angela_Merk~ Ange~ fema~ white     protest~ 1954~
## # i 23 more rows
## # i 5 more variables: death <lgl>, birthplace <chr>, deathplace <chr>,
## #   plain_text <chr>, average_traffic <dbl>
\end{verbatim}

\begin{Shaded}
\begin{Highlighting}[]
\FunctionTok{ggplot}\NormalTok{(all\_countries\_traffic, }\FunctionTok{aes}\NormalTok{(}\AttributeTok{x =}\NormalTok{ sex, }\AttributeTok{y =}\NormalTok{ average\_traffic, }\AttributeTok{color =}\NormalTok{ sex)) }\SpecialCharTok{+}
  \FunctionTok{geom\_boxplot}\NormalTok{() }\SpecialCharTok{+}
  \FunctionTok{facet\_wrap}\NormalTok{(}\SpecialCharTok{\textasciitilde{}}\NormalTok{country, }\AttributeTok{scales =} \StringTok{"free\_y"}\NormalTok{) }\SpecialCharTok{+}
  \FunctionTok{labs}\NormalTok{(}\AttributeTok{title =} \StringTok{"Distribution of Average Traffic per Country and Sex"}\NormalTok{,}
       \AttributeTok{x =} \StringTok{"Sex"}\NormalTok{,}
       \AttributeTok{y =} \StringTok{"Average Traffic"}\NormalTok{) }\SpecialCharTok{+}
  \FunctionTok{scale\_color\_manual}\NormalTok{(}\AttributeTok{values =} \FunctionTok{c}\NormalTok{(}\StringTok{"male"} \OtherTok{=} \StringTok{"blue"}\NormalTok{, }\StringTok{"female"} \OtherTok{=} \StringTok{"pink"}\NormalTok{)) }\SpecialCharTok{+}
  \FunctionTok{theme\_minimal}\NormalTok{() }\SpecialCharTok{+}
  \FunctionTok{theme}\NormalTok{(}\AttributeTok{legend.position =} \StringTok{"none"}\NormalTok{)}
\end{Highlighting}
\end{Shaded}

\begin{verbatim}
## Warning: Removed 1431 rows containing non-finite values (`stat_boxplot()`).
\end{verbatim}

\includegraphics{Data_Report_files/figure-latex/unnamed-chunk-12-1.pdf}

\begin{Shaded}
\begin{Highlighting}[]
\CommentTok{\#zoom in, focussing on lower quartiles, ignoring outliers}
\FunctionTok{ggplot}\NormalTok{(all\_countries\_traffic, }\FunctionTok{aes}\NormalTok{(}\AttributeTok{x =}\NormalTok{ sex, }\AttributeTok{y =}\NormalTok{ average\_traffic, }\AttributeTok{color =}\NormalTok{ sex)) }\SpecialCharTok{+}
  \FunctionTok{geom\_boxplot}\NormalTok{() }\SpecialCharTok{+}
  \FunctionTok{facet\_wrap}\NormalTok{(}\SpecialCharTok{\textasciitilde{}}\NormalTok{country, }\AttributeTok{scales =} \StringTok{"free\_y"}\NormalTok{) }\SpecialCharTok{+}
  \FunctionTok{labs}\NormalTok{(}\AttributeTok{title =} \StringTok{"Distribution of Average Traffic per Country and Sex"}\NormalTok{,}
       \AttributeTok{x =} \StringTok{"Sex"}\NormalTok{,}
       \AttributeTok{y =} \StringTok{"Average Traffic"}\NormalTok{) }\SpecialCharTok{+}
  \FunctionTok{scale\_color\_manual}\NormalTok{(}\AttributeTok{values =} \FunctionTok{c}\NormalTok{(}\StringTok{"male"} \OtherTok{=} \StringTok{"blue"}\NormalTok{, }\StringTok{"female"} \OtherTok{=} \StringTok{"pink"}\NormalTok{)) }\SpecialCharTok{+}
  \FunctionTok{theme\_minimal}\NormalTok{() }\SpecialCharTok{+}
  \FunctionTok{theme}\NormalTok{(}\AttributeTok{legend.position =} \StringTok{"none"}\NormalTok{) }\SpecialCharTok{+}
  \FunctionTok{ylim}\NormalTok{(}\DecValTok{1}\NormalTok{, }\DecValTok{1500}\NormalTok{)  }\CommentTok{\# Adjust the y{-}axis limits}
\end{Highlighting}
\end{Shaded}

\begin{verbatim}
## Warning: Removed 3356 rows containing non-finite values (`stat_boxplot()`).
\end{verbatim}

\includegraphics{Data_Report_files/figure-latex/unnamed-chunk-12-2.pdf}

\begin{Shaded}
\begin{Highlighting}[]
\CommentTok{\# ggplot(all\_countries\_traffic, aes(x = log(average\_traffic), color = sex)) +}
\CommentTok{\#   geom\_density(size = 1) +}
\CommentTok{\#   facet\_wrap(\textasciitilde{} country, scales = "free\_y") +}
\CommentTok{\#   labs(title = "Density of Average Traffic per Country and Sex",}
\CommentTok{\#        x = "Average Traffic",}
\CommentTok{\#        y = "Density",}
\CommentTok{\#        color = "Sex") +}
\CommentTok{\#   scale\_color\_manual(values = c("pink", "blue")) +}
\CommentTok{\#   theme\_minimal()}
\end{Highlighting}
\end{Shaded}

Now, let's have a look at the word counts for female and male
politicians (for the further analysis, this data will be matched on
confounding aspect of fame)

\begin{Shaded}
\begin{Highlighting}[]
\CommentTok{\#get word count}

\NormalTok{all\_countries}\SpecialCharTok{$}\NormalTok{word\_count }\OtherTok{\textless{}{-}} \FunctionTok{sapply}\NormalTok{(all\_countries}\SpecialCharTok{$}\NormalTok{plain\_text, count\_words)}


\NormalTok{avg\_word\_count }\OtherTok{\textless{}{-}}\NormalTok{ all\_countries }\SpecialCharTok{\%\textgreater{}\%}
  \FunctionTok{group\_by}\NormalTok{(country, sex) }\SpecialCharTok{\%\textgreater{}\%}
  \FunctionTok{summarise}\NormalTok{(}\AttributeTok{avg\_word\_count =} \FunctionTok{mean}\NormalTok{(word\_count))}
\end{Highlighting}
\end{Shaded}

\begin{verbatim}
## `summarise()` has grouped output by 'country'. You can override using the
## `.groups` argument.
\end{verbatim}

\begin{Shaded}
\begin{Highlighting}[]
\FunctionTok{ggplot}\NormalTok{(avg\_word\_count, }\FunctionTok{aes}\NormalTok{(}\AttributeTok{x =}\NormalTok{ sex, }\AttributeTok{y =}\NormalTok{ avg\_word\_count, }\AttributeTok{fill =}\NormalTok{ sex)) }\SpecialCharTok{+}
  \FunctionTok{geom\_bar}\NormalTok{(}\AttributeTok{stat =} \StringTok{"identity"}\NormalTok{, }\AttributeTok{position =} \StringTok{"dodge"}\NormalTok{) }\SpecialCharTok{+}
  \FunctionTok{facet\_wrap}\NormalTok{(}\SpecialCharTok{\textasciitilde{}}\NormalTok{ country, }\AttributeTok{scales =} \StringTok{"free\_y"}\NormalTok{) }\SpecialCharTok{+}
  \FunctionTok{labs}\NormalTok{(}\AttributeTok{title =} \StringTok{"Average Word Count per Sex in Each Country"}\NormalTok{,}
       \AttributeTok{x =} \StringTok{"Sex"}\NormalTok{,}
       \AttributeTok{y =} \StringTok{"Average Word Count"}\NormalTok{) }\SpecialCharTok{+}
  \FunctionTok{scale\_fill\_manual}\NormalTok{(}\AttributeTok{values =} \FunctionTok{c}\NormalTok{(}\StringTok{"male"} \OtherTok{=} \StringTok{"blue"}\NormalTok{, }\StringTok{"female"} \OtherTok{=} \StringTok{"pink"}\NormalTok{)) }\SpecialCharTok{+}
  \FunctionTok{theme\_minimal}\NormalTok{() }\SpecialCharTok{+}
  \FunctionTok{theme}\NormalTok{(}\AttributeTok{legend.position =} \StringTok{"none"}\NormalTok{)}
\end{Highlighting}
\end{Shaded}

\includegraphics{Data_Report_files/figure-latex/unnamed-chunk-13-1.pdf}

\end{document}
